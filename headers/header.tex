\usepackage{fontspec}
\usepackage[english]{babel}

\usepackage{fontawesome5}

\usepackage{amsmath}
\usepackage{mathtools}
\usepackage{amssymb}
\usepackage{mleftright}

\usepackage[
  math-style=ISO,
  bold-style=ISO,
  sans-style=italic,
  nabla=upright,
  partial=upright,
  mathrm=sym,
]{unicode-math}
\setmathfont{Fira Math}[Scale=MatchLowercase]

\usepackage[
  locale=UK,
  separate-uncertainty=true,
  per-mode=symbol-or-fraction,
]{siunitx}

\usepackage[
  german=quotes,
  autostyle,
]{csquotes}
\usepackage{xfrac}

\usepackage{tabularray}
\UseTblrLibrary{booktabs, siunitx, varwidth}
\usepackage{threeparttable}

\usepackage{graphicx}

\usepackage[
  compatibility=false,
]{caption}
\usepackage{subcaption}

\usepackage{xcolor}
\usepackage{metalogo}
\usepackage{pdflscape}

\usepackage{fancyvrb}
\usepackage[outputdir=build]{minted2}
\setminted{
  autogobble,
  breaklines,
  stripnl=true,
}
\usemintedstyle{code}

\usepackage[
  theorems,
  many,
]{tcolorbox}

\usepackage{adjustbox}

\usepackage{tikz}
\usetikzlibrary{
  arrows,
  arrows.meta,
  graphs,
  graphdrawing,
  positioning,
  shadows,
  shapes,
}

\usepackage[
  shortcuts,
]{extdash}

\usepackage[noframe]{showframe}
\usepackage{bookmark}


\tikzset{
    position label/.style={
       below = 3pt,
       text height = 1.5ex,
       text depth = 1ex
    },
    brace/.style={
     decoration={brace},
     decorate
   }
}

% This adds a circle with a picture of your choice in it.
% Usage:
% \roundpic[<optional arguments>, e.g. xshift or yshift]\
% {<radius of the cirlce [cm]>}{<picture width [cm]>}{<path_to_picture>}{x pos}{y pos}{label}
% TO BE PUT INSIDE TIKZ ENVIRONMENT (i.e. \begin{tikzpicture} \roundpic... \end{tikzpicture})
\NewDocumentCommand \roundpic {o o m m m m m o}{%
  \node [%
    circle, draw, minimum size=#3, #1,
    path picture = {%
      \node [#2] at (path picture bounding box.center) {%
        \IfNoValueF{#8}{\hyperlink{#8}\begingroup}
        \includegraphics[#4]{#5}
        \IfNoValueF{#8}{\endgroup}
      };
    }
  ] at (#6, #7) {};
}%

\NewDocumentCommand\sectionslide{m o}{%
  {
  \setbeamertemplate{background} {
    \begin{tikzpicture}[remember picture,overlay]
      \node [anchor=north, yshift=0.25cm] at (current page.north) {
        \reflectbox{\includegraphics[height=\pageheight+0.5cm]{images/sec_img.jpg}}
      };
      \fill[black, path fading=titlefade, fading transform={xscale=.3, xshift=-1.25cm}]
          (current page.north west) ++(0cm,1cm) rectangle (current page.south east);
    \end{tikzpicture}
  }
  \begin{frame}
    \begin{tikzpicture}[remember picture,overlay]
      \node [anchor=east, opacity=0.3] at ([xshift=-15pt]current page.east) {#2};
      \node [font=\bfseries\huge] (title) at (current page.center) {#1};
      \draw [tugreen, line width=1.2pt] (title.south west) -- (title.north west);
    \end{tikzpicture}
  \end{frame}
  }
}

