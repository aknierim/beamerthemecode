\begin{frame}{Note}
  About this theme:
  \begin{itemize}
    \item To install this theme, at least \texttt{beamerthemecode.sty} has to be moved to a directory that is searchable by \LaTeX.
      This can be one of
      \begin{itemize}
        \item \texttt{TEXMFHOME/tex/latex/beamerthemecode}. \texttt{TEXMFHOME} can be accessed via \texttt{kpsewhich --var-value TEXMFHOME};
          usually this returns \texttt{\$HOME/texmf}.
        \item The same directory you compile your document in.
        \item Any directory which is included in \texttt{TEXINPUTS}.
      \end{itemize}
  \end{itemize}

  You can install the theme with the following command:\\
  \texttt{\footnotesize\$ cd `kpsewhich --var-value TEXMFHOME` \&\& git clone https://github.com/aknierim/beamerthemecode}
\end{frame}


\begin{splitframe}[fragile]{Split Frame}{Code Example}
  \begin{columns}[t]
    \begin{column}{0.58\textwidth}
      This theme allows you to create visually split slides.
      \begin{itemize}
        \setlength{\itemsep}{1em}
        \item Use the \mintinline{latex}{\splitframe} command inside a frame
          to create the split slide
        \item Using curly braces, you can also set an optional second title
        \item An optional argument following the titles changes the split ratio
        \item The dark half of the slide is intended mainly for code
          examples together with the provided pygments style
      \end{itemize}
    \end{column}
    \hfill
    \begin{column}{0.38\textwidth}
      \footnotesize
      \color{cwhite}
      \inputminted{latex}{content/example_split_frame.tex}
    \end{column}
  \end{columns}
\end{splitframe}

\begin{splitframe}[fragile]{Split Frame, split ratio=0.5}{Code Example}{0.5}
  \begin{columns}[t]
    \begin{column}{0.48\textwidth}
      \begin{itemize}
        \setlength{\itemsep}{1em}
        \item The \mintinline{latex}+{⟨split ratio⟩}+ option can be used to change
          the split ratio
        \item This is useful for code examples that require more width
        \item The default value is \texttt{0.6}
      \end{itemize}
    \end{column}
    \hfill
    \begin{column}{0.48\textwidth}
      \color{cwhite}
      \inputminted{latex}{content/example_split_frame_var.tex}
    \end{column}
  \end{columns}
\end{splitframe}

\begin{darkframe}[fragile]{Dark Frame}
  For larger code examples, the \mintinline{latex}{darkframe} environment can be used:
  \begin{center}
    \inputminted{latex}{content/example_dark_frame.tex}
  \end{center}
\end{darkframe}
