\begin{frame}{Note}
  About this theme:
  \begin{itemize}
    \item To install this theme, at least \texttt{beamerthemecode.sty} has to be moved to a directory that is searchable by \LaTeX.
      This can be one of
      \begin{itemize}
        \item \texttt{TEXMFHOME/tex/latex/beamerthemecode}. \texttt{TEXMFHOME} can be accessed via \texttt{kpsewhich --var-value TEXMFHOME};
          usually this returns \texttt{\$HOME/texmf}.
        \item The same directory you compile your document in.
        \item Any directory which is included in \texttt{TEXINPUTS}.
      \end{itemize}
  \end{itemize}

  You can install the theme with a one-liner:\\
  \texttt{\footnotesize\$ cd `kpsewhich --var-value TEXMFHOME` \&\& git clone https://github.com/aknierim/beamerthemecode}
\end{frame}


\begin{frame}[split, fragile]{Split Frame}
  \framesubtitle{Code Example}
  \begin{columns}[t]
    \begin{column}{0.58\textwidth}
      This theme allows you to create visually split slides.
      \begin{itemize}
        \setlength{\itemsep}{1em}
        \item Use the \texttt{split} key with the \texttt{frame} environment
          to create the split slide
        \item Set an optional second title using the
          \mintinline{latex}+\framesubtitle+ command
        \item The dark half of the slide is intended mainly for code
          examples together with the provided pygments style
      \end{itemize}
    \end{column}
    \hfill
    \begin{column}{0.38\textwidth}

    \end{column}
  \end{columns}
\end{frame}

\begin{frame}[split=0.4]{Split Frame}

\end{frame}

\begin{frame}[full,fragile]{Dark Frame}
\end{frame}
